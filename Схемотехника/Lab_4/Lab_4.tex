\documentclass[a4paper,14pt]{extarticle} 
\usepackage[a4paper,top=1.5cm, bottom=1.5cm, left=2cm, right=1cm]{geometry}
%\usepackage[T2A]{fontenc}
%\usepackage[english, russian]{babel}
\usepackage{graphicx}
\graphicspath{{./pics/}}
\DeclareGraphicsExtensions{.pdf,.png,}
\usepackage[table]{xcolor}

\usepackage{fontspec}
\setmainfont{Times New Roman}
\setsansfont{FreeSans}
\setmonofont{FreeMono}
\renewcommand{\baselinestretch}{1.5}
\usepackage{polyglossia}
\setdefaultlanguage{russian}
\setotherlanguages{english,russian}
\usepackage{setspace}
\usepackage[many]{tcolorbox}
\usepackage{array}
\usepackage{longtable}

\begin{document}
    \begin{center}
        \thispagestyle{empty}
        \begin{singlespace}
        ФЕДЕРАЛЬНОЕ АГЕНТСТВО СВЯЗИ

        ФЕДЕРАЛЬНОЕ ГОСУДАРСТВЕННОЕ БЮДЖЕТНОЕ ОБРАЗОВАТЕЛЬНОЕ

        УЧРЕЖДЕНИЕ ВЫСШЕГО ОБРАЗОВАНИЯ

        «САНКТ-ПЕТЕРБУРГСКИЙ ГОСУДАРСТВЕННЫЙ УНИВЕРСИТЕТ ТЕЛЕКОММУНИКАЦИЙ ИМ. ПРОФ. М.А. БОНЧ-БРУЕВИЧА»

        (СПбГУТ)
        \end{singlespace}
        \vspace{-1ex}
        \rule{\textwidth}{0.4pt}
        \vspace{-5ex}

        \vspace{100px}
        \textbf{Лабораторная работа №4}\\
        Исследование свойств модели резисторного каскада с общей базой в частотной и временной областях на ПК

    \vspace{100px}
    \end{center}
    \vspace{4ex}
    \begin{flushright}
    \parbox{12 cm}{
    \begin{flushleft}
        Выполнила бригада:\\
        Группа ИКТЗ-83\\
        \underline{Громов А.А., Миколаени М.С., Мазеин Д.С.} \hfill \rule[-0.85ex]{0.09\textwidth}{0.6pt}\\
        \vspace{-1ex}
        \footnotesize \textit{ (Ф.И.О., № группы) \hfill (подпись)} \normalsize


    \end{flushleft}
    }
    \end{flushright}
    \begin{center}
        \vfill
        Санкт-Петербург

        2020

    \end{center}
    \newpage

    \textbf{Цель работы:} Изучить свойства усилительного каскада с общим коллектором (ОК) в режиме малого сигнала. Выполнить анализ в частотной и временной областях. Исследовать свойства каскада при изменении сопротивлений источника сигнала, нагрузки и элементов схемы. Определить входное и выходное сопротивления каскада.

    % \textbf{\emph{Объектом исследования}} является схема усилительного каскада на
    % биполярном транзисторе с общим коллектором. По определению в схеме с ОК
    % коллектор транзистора присоединяется к проводу, общему для входа и выхода
    % каскада. На рис. 1, а показано простейшее изображение схемы с ОК.
    
    % \begin{center}
    %     \includegraphics[scale=0.85]{0.1.png}
    % \end{center}

    % Другое представление транзистора с ОК показано на рис. 1, б. Такое
    % изображение каскада с ОК позволяет рассматривать его как структуру с ОЭ,
    % охваченную ОС. В этой схеме имеет место последовательная по входу и
    % параллельная по выходу отрицательная ОС (рис. 1, в). Полная принципиальная
    % схема каскада ОК представлена на рис. 2)

    % \begin{center}
    %     \includegraphics[scale=0.85]{0.2.png}
    % \end{center}

    % Переменная составляющая сигнала поступает на базу транзистора через
    % разделительный конденсатор $C_{\text{Р1}}$ , а передается в нагрузку $R_{2H}$ через
    % разделительный конденсатор $C_{P2}$ из эмиттера. Сопротивление источника
    % питания $E_0$ переменному току практически равно нулю, поэтому коллектор
    % оказывается соединенным с общим проводом и схема соответствует структуре
    % соединений на рис. 1, б. В схеме с ОК, как правило, не включают в
    % коллекторную цепь резистор $R_K$ . В этом случае всё напряжение питания
    % делится между транзистором и $R_{\text{Э}}$ . При питании транзистора с эмиттерной
    % стабилизацией, применённой здесь, ток покоя изменяется незначительно, а
    % увеличение напряжений между электродами транзистора ($U_{\text{КЭ}}$ и особенно $U_{\text{КБ}}$)
    % снижает значение ёмкости $C_K$ . Этот факт и отсутствие эффекта Миллера (при
    % $R_K$ = 0)дают основания для сохранения модели транзистора неизменной.

    % В работе используются данные лабораторного макета, при этом
    % сохраняются номинальные значения всех элементов схемы, напряжение
    % питания, ток покоя, транзистор КТ503В, согласно рис.2.1 [1]. В этой работе
    % моделируется усилитель на основе реального действующего макета.
    % Эквивалентная схема усилителя с ОК для переменного тока изображена на
    % рис. 3. Входной сигнал поступает через разделительный конденсатор $C_{\text{Р1}}$ на
    % базу транзистора (узел 2). Элемент $R_{\text{б}}$ отражает эквивалентное сопротивление
    % базового делителя – параллельное включение $R_{\text{б1}}$ и $R_{\text{б2}}$ . Выходной сигнал через
    % разделительный конденсатор $C_{Р2}$ подаётся в нагрузку $R_{\text{2н}}$, $C_{\text{2н}}$ (узел 7)
    % Коллектор транзистора заземлён непосредственно и является общим для входа
    % и выхода усилителя. Другие элементы эквивалентной схемы соответствуют
    % приведённым на рис. 3 и соответствуют параметрам лабораторного макета.
    % Значения внутренних ёмкостей транзистора и его средний коэффициент
    % усиления тока $h_{21}$ взяты из справочных данных. В схему введено выходное
    % сопротивление транзистора 1/ $h_{22}$ . Оно определяется током покоя $I_{0K}$ = 4 мА и
    % ориентировочным значением напряжения Эрли, равным 100 В. При токе
    % коллектора $I_{0K}$ = 4 мА и $h_{21}$ = 75 в базе транзистора протекает ток $I_{\text{0Б}}$ = 53 мкА.
    % Принимая $U_T$ = 25.6 мВ, получаем сопротивление перехода база-эмиттер
    % rбэ = 470 Ом.

    % \begin{center}
    %     \includegraphics[scale=0.9]{0.3.png}
    % \end{center}

    \newpage
    \textbf{Пункт 1:}
    \begin{center}
        \includegraphics[scale=0.25]{1.jpg}
    \end{center}
    \begin{center}
        Входное сопротивление с учетом и без учета резистора $R_{\text{э}}$ 
    \end{center}
    \begin{table}[ht]
        \begin{center}
            \caption{Измерение входного сопротивления каскада с ОК}
            \begin{tabular}{ |c|c| }
                \hline
                Измерение & Величина входного сопротивления, КОм\\
                \hline
                с учётом сопротивления $R_{\text{э}}$ & 10,6 МОм\\
                \hline
                без учёта сопротивления $R_{\text{э}}$ & 19,2 Ом\\
                \hline
            \end{tabular}
        \end{center}
    \end{table}

    \newpage
    \textbf{Пункт 2:}
    \begin{center}
        \includegraphics[scale=0.25]{2.jpg}
    \end{center}
    \begin{center}
        Выходное сопротивление транзистора и каскада
    \end{center}

    Выходное сопротивление с учетом $R_{\text{э}}$ 1,994 кОм

    Выходное сопротивление без учета $R_{\text{э}}$ 740,15 кОм

    \textbf{Выводы по пункту 2:}
    \vspace{-6ex}
    \begin{singlespace}
        \begin{itemize}
            \item Входное сопротивление каскада с ОК примерно на 3 порядка больше, чем выходное.
        \end{itemize}
    \end{singlespace}

    \newpage 
    \textbf{Пункт 3:}
    \begin{center}
        \includegraphics[scale=0.25]{3.jpg}
    \end{center}
   \begin{center}
        АЧХ и ФЧХ каскада
    \end{center}

    \begin{table}[ht]
        \begin{center}
            \caption{Измерение АЧХ каскада с ОБ}
            \begin{tabular}{ |c|c|c|c|c| }
                \hline
                $K_{\text{скв}}$, Дб & ($K_{\text{скв}}$ - 3), Дб&fн, Гц & fв, МГц & ∆f=fв−fн, МГц \\
                \hline 
                1.84 & -1.16 & 157.45 & 1,11 & 1.1098 \\
                \hline
            \end{tabular}\\
        \end{center}
    \end{table}
    \textbf{Выводы по пункту 3:}
    \vspace{-6ex}
    \begin{singlespace}
        \begin{itemize}
            \item Схема с ОК не инвертирует входной сигнал.
            \item У схемы каскада с ОК рабочая полоса частот больше, чем у схемы с ОЭ.
            \item Схема каскада с ОК, в отличие от схемы касакада с ОЭ, ослабляет сигнал. 
        \end{itemize}
    \end{singlespace}


    \newpage
    \textbf{Пункт 4:}
    \vspace{-1cm}
    \begin{table}[h!]
        \small
        \begin{center}
            \caption{Измерение ПХ каскада с ОК}
            \begin{tabular}{|>{\centering}m{7.6cm}|>{\centering}m{8.5cm}|}
                \hline 
                \rowcolor{gray} Время импульса & $t_{\text{и}}$= 25 мкс 
                \tabularnewline
                \hline
                Частота f, Гц & 20000
                \tabularnewline
                \hline 
                Осциллограмма импульса & \begin{center}\includegraphics[scale=0.08]{4.1.jpg}\end{center}
                \tabularnewline
                \hline 
                \parbox[c][3cm]{7.6cm}{
                    Измеренный спад вершины импульса ∆, \% 
                    \begin{center}
                        $∆=\frac{U_{\text{уст}}-U_{\text{вых}}}{U_{\text{уст}}} \cdot 100\%$
                    \end{center} 
                    }& 2.35
                \tabularnewline
                \hline 
                Рассчитанный спад вершины импульса ∆, \% & 2.47 
                \tabularnewline
                \hline 
                \multicolumn{2}{|c|}{Осциллограмма увеличенной области нарастания импульса} 
                \tabularnewline
                \hline 
                \multicolumn{2}{|c|}{\parbox[c]{4cm}{\includegraphics[scale=0.08]{4.2.jpg}}}
                \tabularnewline
                \hline
                \parbox[c][3cm]{7.6cm}{Измеренное время нарастания импульса \\ $t_{\text{Н}}$ = $t_{\text{2}}$ – $t_{\text{1}}$, нс} & 312.8
                \tabularnewline
                \hline 
                Рассчитанное время нарастания импульса $t_{\text{н}}$, нс & 315.3
                \tabularnewline
                \hline 
           \end{tabular}
        \end{center}
    \end{table}

    \vspace{-1.2cm}
    \begin{table}[h!]
        \small
        \begin{center}
            \begin{tabular}{|>{\centering}m{7.6cm}|>{\centering}m{8.5cm}|}
                \hline
                \rowcolor{gray} Время импульса & $t_{\text{и}}$= 1.25 мс 
                \tabularnewline
                \hline
                Частота f, Гц & 400
                \tabularnewline
                \hline 
                \multicolumn{2}{|c|}{Осциллограмма импульса}
                \tabularnewline
                \hline
                \multicolumn{2}{|c|}{\parbox[c]{6cm}{\includegraphics[scale=0.08]{4.2.jpg}}}
                \tabularnewline
                \hline
                \parbox[c][2cm]{7.6cm}{
                    Измеренный спад вершины импульса ∆, \% 
                    \begin{center}
                        $∆=\frac{U_{\text{уст}}-U_{\text{вых}}}{U_{\text{уст}}} \cdot 100\%$
                    \end{center} 
                    }& 71
                \tabularnewline
                \hline 
                Рассчитанный спад вершины импульса ∆, \% & 124
                \tabularnewline
                \hline
            \end{tabular}
        \end{center}
    \end{table}

    \textbf{Выводы по пункту 4:}
    \vspace{-6ex}
    \begin{singlespace}
        \begin{itemize}
            \item Измеренный спад вершины импульса практически совпадает с рассчитанным спадом вершины импульса;
            \item Измеренное время нарастания импульса практически совпадает с рассчитанное временем нарастания импульса;
        \end{itemize}
    \end{singlespace}

    \newpage
    \textbf{Пункт 5}
    \begin{table}[ht]
        \begin{center}
            \caption{Оценка влияния параметров схемы на ПХ и АЧХ}
            \begin{tabular}{|c|c|c|c|c|c|c|c|}
                \hline 
                № & $R_1$ & $R_2$ & $K_{\text{скв}}$ & $f_{\text{н}}$ & $f_{\text{в}}$ & \parbox[c]{4cm}{\begin{center}∆ \\при $t_{\text{и}}$= 25 мкс \end{center}} & \parbox[c]{4cm}{\begin{center}$t_{n}$ \\при $t_{\text{и}}$= 1.25 мс \end{center}}
                \tabularnewline
                \hline 
                п/п & кОм & кОм & дБ & Гц & МГц & \% & нс
                \tabularnewline
                \hline 
                1 & 1 & 3,6 & -2,7 & 41 & 18,6 & 27.9 & 19.12
                \tabularnewline
                \hline 
                2 & 1 & 10 & -2,7 & 41 & 18,6 & 27.2 & 19.3
                \tabularnewline
                \hline        
                3 & 5 & 3,6 & -8.8 & 13.8 & 11 & 15 & 46.44
                \tabularnewline
                \hline        
                4 & 5 & 10 & -8,8 & 13,6 & 11,1 & 13.8 & 46.67 
                \tabularnewline
                \hline        
            \end{tabular}
        \end{center}
    \end{table}

    % \begin{center}
    %     \includegraphics[scale=0.3]{5.5.jpg}
    % \end{center}

    % АЧХ при $R_2$ = 3.6кОм

    % \begin{center}
    %     \includegraphics[scale=0.3]{5.6.jpg}
    % \end{center}

    % АЧХ при $R_2$ = 10кОм

    % \begin{center}
    %     \includegraphics[scale=0.3]{5.1.jpg}
    % \end{center}

    % ПХ при $R_2$ = 3.6кОм, f = 20кГц

    % \begin{center}
    %     \includegraphics[scale=0.3]{5.2.jpg}
    % \end{center}

    % ПХ при $R_2$ = 10ОкОм, f=20кГц

    % \begin{center}
    %     \includegraphics[scale=0.3]{5.3.jpg}
    % \end{center}
    
    % ПХ при f = 400Гц, $R_2$ = 10кОм 

    % \begin{center}
    %     \includegraphics[scale=0.3]{5.4.jpg}
    % \end{center}

    ПХ при f = 400Гц, $R_2$ = 3.6кОм 

    \textbf{Выводы по пункту 5:}
    \vspace{-6ex}
    \begin{singlespace}
        \begin{itemize}
           \item Увеличение $R_1$ уменьшает измеренный спад вершины импульса и увеличивает 
           $t_{\text{и}}$, а увеличение $R_2$ практически не оказывает эффекта на эти параметры 
           \item Увеличение $R_1$ уменьшает $K_{\text{скв}}$, а также сдвигает вниз по 
           частоте $f_{\text{н}}$ и $f_{\text{в}}$ и уменьшает рабочий диапазон частот. 
           Увеличение $R_2$ незначительно влияет на эти параметры.
        \end{itemize}
    \end{singlespace}
\end{document}