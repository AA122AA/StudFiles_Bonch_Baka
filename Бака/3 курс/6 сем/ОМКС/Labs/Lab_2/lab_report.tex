\documentclass[a4paper,14pt]{extarticle} 
\usepackage[a4paper,top=1.5cm, bottom=1.5cm, left=2cm, right=1cm]{geometry}
%\usepackage[T2A]{fontenc}
%\usepackage[english, russian]{babel}
\usepackage{graphicx}
\DeclareGraphicsExtensions{.pdf,.png,.jpg}

\usepackage{fontspec}
\setmainfont{Times New Roman}
\setsansfont{FreeSans}
\setmonofont{FreeMono}
\renewcommand{\baselinestretch}{1.5}
\usepackage{polyglossia}
\setdefaultlanguage{russian}
\setotherlanguages{english,russian}
\usepackage{setspace}
\usepackage[many]{tcolorbox}
\usepackage{listings}
\usepackage{xcolor}

\definecolor{codegreen}{rgb}{0,0.6,0}
\definecolor{codegray}{rgb}{0.5,0.5,0.5}
\definecolor{codepurple}{rgb}{0.58,0,0.82}
\definecolor{backcolour}{rgb}{0.95,0.95,0.92}

\lstdefinestyle{mystyle}{
    backgroundcolor=\color{backcolour},   
    keywordstyle=\color{magenta},
    numberstyle=\tiny\color{codegray},
    stringstyle=\color{codepurple},
    basicstyle=\ttfamily\footnotesize,
    breakatwhitespace=false,         
    breaklines=true,                 
    captionpos=b,                    
    keepspaces=true,                 
    numbers=left,                    
    numbersep=5pt,                  
    showspaces=false,                
    showstringspaces=false,
    showtabs=false,                  
    tabsize=2
}

\lstset{style=mystyle}

\begin{document}
    \begin{center}
        \thispagestyle{empty}
        \begin{singlespace}
        ФЕДЕРАЛЬНОЕ АГЕНТСТВО СВЯЗИ

        ФЕДЕРАЛЬНОЕ ГОСУДАРСТВЕННОЕ БЮДЖЕТНОЕ ОБРАЗОВАТЕЛЬНОЕ

        УЧРЕЖДЕНИЕ ВЫСШЕГО ОБРАЗОВАНИЯ

        «САНКТ-ПЕТЕРБУРГСКИЙ ГОСУДАРСТВЕННЫЙ УНИВЕРСИТЕТ ТЕЛЕКОММУНИКАЦИЙ ИМ. ПРОФ. М.А. БОНЧ-БРУЕВИЧА»

        (СПбГУТ)
        \end{singlespace}
        \vspace{-1ex}
        \rule{\textwidth}{0.4pt}
        \vspace{-5ex}

        Факультет \underline{Инфокоммуникационных сетей и систем}

        Кафедра \underline{Защищенных систем связи}
        \vspace{10ex}

        \textbf{Лабораторная работа №2}\\
        


    \end{center}
    \vspace{4ex}
    \begin{flushright}
    \parbox{10 cm}{
    \begin{flushleft}
        Выполнили студенты группы ИКТЗ-83:

        \underline{Громов А.А., Миколаени М.С., Мазеин Д.С.} \hfill \rule[-0.85ex]{0.1\textwidth}{0.6pt}

        \footnotesize \textit{ (Ф.И.О., № группы) \hfill (подпись)} \normalsize

        Проверил:

        \underline{Скорых М.А.} \hfill \rule[-0.85ex]{0.1\textwidth}{0.6pt}

        (\footnotesize \textit{уч. степень, уч. звание, Ф.И.О.) \hfill (подпись)} \normalsize

    \end{flushleft}
    }
    \end{flushright}
    \begin{center}
        \vfill
        Санкт-Петербург

        2021

    \end{center}
    \newpage

    \textbf{Цель лабораторной работы:}
    Повторение основных концепций из курса «Основы построения компьютерных сетей
    

    \begin{enumerate}
        \item \textbf{Настройка маршрутизатора}
        \begin{itemize}
                \item \textbf{Настройте основные параметры устройства.}
                \begin{enumerate}
                        \item \textbf{Имя узла, как показано в таблице адресации.} 
                                \begin{lstlisting}[language=bash]
Router1(config)#hostname RTA
                                \end{lstlisting}
                        \item \textbf{Установите Ciscoenpa55 в качестве зашифрованного пароля.} 
                                \begin{lstlisting}
RTA(config)#enable secret Ciscoenpa55 
                                \end{lstlisting}
                        \item \textbf{Установите Ciscolinepa55 в качестве пароля на виртуальные терминалы (линии).} 
                                \begin{lstlisting}
RTA(config)#line vty 0 15
RTA(config-line)#password Ciscolinepa55 
                                \end{lstlisting}
                        \item \textbf{Настройте баннер MOTD (сообщения дня).} 
                                \begin{lstlisting}
RTA(config)#banner motd @enter password@
                                \end{lstlisting}
               \end{enumerate}
               \item \textbf{Настройка параметров интерфейса.}
                        \begin{enumerate}
                            \item \textbf{Адресация}
                                \begin{lstlisting}
RTA(config)#int g0/0/0
RTA(config-int)#ip address 10.10.10.1 255.255.255.0
                                \end{lstlisting}
                            \item \textbf{Описания интерфейса.}
                                \begin{lstlisting}
RTA(config-int)#description int to SW1
RTA(config-int)#no shutdown
                                \end{lstlisting}
                            \item \textbf{Cохраните конфигурацию.}
                                \begin{lstlisting}
RTA#copy run st
                                \end{lstlisting}
                        \end{enumerate}
        \end{itemize}
        \item \textbf{Настройка коммутаторов SW1 и SW2.}
        \begin{itemize}
                \item \textbf{Настройте интерфейс управления по умолчанию таким образом, чтобы он принимал подключения по сети от локальных и удаленных узлов. Используйте значения из таблицы адресации.}
                \begin{lstlisting}
SW1(config)#int vlan 1
SW1(config-int)#ip address 10.10.10.2 255.255.255.0
SW1(config-int)#no shutdown
                \end{lstlisting}
                \item \textbf{Настройте зашифрованный пароль}
                \begin{lstlisting}
SW1(config)#enable secret cisco
                \end{lstlisting}
                \item \textbf{Настройте коммутаторы таким образом, чтобы они могли отправлять данные узлам в удаленных сетях.}
                \begin{lstlisting}
SW1(config)#ip default-gateway 10.10.10.1
                \end{lstlisting}
                \item \textbf{Cохраните конфигурацию.}
                    \begin{lstlisting}
RTA#copy run st
                    \end{lstlisting}
        \end{itemize}
    \end{enumerate}
    
    \textbf{running config on RTA}
    \begin{lstlisting}
no service timestamps debug datetime msec
no service password-encryption
!
hostname RTA
!
enable secret 5 $1$mERr$Amm/da5NtiazLuZDbgqZ60
!
ip cef
no ipv6 cef
!
spanning-tree mode pvst
!
interface GigabitEthernet0/0/0
 description int to SW1
 ip address 10.10.10.1 255.255.255.0
 duplex auto
 speed auto
!
interface GigabitEthernet0/0/1
 description int to SW2
 ip address 10.10.20.1 255.255.255.0
 duplex auto
 speed auto
!
interface GigabitEthernet0/0/2
 no ip address
 duplex auto
 speed auto
 shutdown
!
interface Vlan1
 no ip address
 shutdown
!
ip classless
!
ip flow-export version 9
!
banner motd ^Center password^C
!
line con 0
!
line aux 0
!
line vty 0 4
 password Ciscolinepa55
 login
line vty 5 15
 password Ciscolinepa55
 login
!
end 
    \end{lstlisting}
\newpage
    \textbf{running config on SW1}
    \begin{lstlisting}
hostname SW1
!
enable secret 5 $1$mERr$hx5rVt7rPNoS4wqbXKX7m0
!
spanning-tree mode pvst
spanning-tree extend system-id
!
interface FastEthernet0/1
.
.
.
interface FastEthernet0/24
!
interface GigabitEthernet0/1
!
interface GigabitEthernet0/2
!
interface Vlan1
 ip address 10.10.10.2 255.255.255.0
!
ip default-gateway 10.10.10.1
!
line con 0
!
line vty 0 4
 login
line vty 5 15
 login
!
end
    \end{lstlisting}
\newpage
    \textbf{running config on SW2}
    \begin{lstlisting}
hostname SW2
!
enable secret 5 $1$mERr$hx5rVt7rPNoS4wqbXKX7m0
!
spanning-tree mode pvst
spanning-tree extend system-id
!
interface FastEthernet0/1
.
.
.
interface FastEthernet0/24
!
interface GigabitEthernet0/1
!
interface GigabitEthernet0/2
!
interface Vlan1
 ip address 10.10.20.2 255.255.255.0
!
ip default-gateway 10.10.20.1
!
line vty 0 4
 login
line vty 5 15
 login
!
end
    \end{lstlisting}
\end{document}